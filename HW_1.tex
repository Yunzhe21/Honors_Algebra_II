\documentclass[12pt]{article}
\usepackage[margin=1in]{geometry}
\usepackage[all]{xy}


\usepackage{amsmath,amsthm,amssymb,color,latexsym}
\usepackage{geometry}        
\geometry{letterpaper}    
\usepackage{graphicx}

\newtheorem{problem}{Problem}

\newenvironment{solution}[1][\it{Solution}]{\textbf{#1. } }{$\square$}


\begin{document}
\noindent Honors Algebra II \hfill Assignment 1\\
Yunzhe Zheng. (2025/02/05)

\hrulefill

\begin{problem}
Prove or disprove $\mathbb{Z}/2\mathbb{Z} \times \mathbb{Z}/2\mathbb{Z}\simeq \mathbb{Z} / 4\mathbb{Z}$.
\end{problem}

\textbf{Disprove:} For every element in $\mathbb{Z}/2\mathbb{Z}\times\mathbb{Z}/2\mathbb{Z}$, say $z = (\overline{a}, \overline{b})$, then $z^{2} = \overline{e} = (\overline{0}, \overline{0})$, so there is no element of order $4$ in $\mathbb{Z}/2\mathbb{Z}\times\mathbb{Z}/2\mathbb{Z}$. However, $\overline{1}\in \mathbb{Z}/4\mathbb{Z}$ has order $4$. Thus they cannot be isomorphic to each other.

\begin{problem}
Assume that the element $a, b, c$ of a group $G$ satisfies $abc = e$, where $e$ is the neutral element. \\
\indent a). does that imply that $bca = e$? \\
\indent b). does that imply that $bac = e$? 

\end{problem}

\textbf{Proof}: \\
\indent a). Since $abc = e$, we have $a = c^{-1}b^{-1}$, thus $bca = bcc^{-1}b^{-1}=e$. \\
\indent b). Consider the group $S_{3}$, and let $a = (1, 2)$, $b = (2, 3)$, and $c = (3, 2, 1)$, then $abc = e$. However, $bac = (2, 3)(1, 2)(3, 2, 1)\neq e$, for it permutes $3$ to $1$ rather than $3$. \qed

\begin{problem}
Determine the number of elements of order 2 in the symmetric group $S_{4}$.
\end{problem}

\textbf{Solution:} They are $(1, 2)$, $(1, 3)$, $(1, 4)$, $(2, 3)$, $(2, 4)$, $(3, 4)$, $(1, 2)(3, 4)$, $(1, 3)(2, 4)$, $(1, 4)(2, 3)$.

\begin{problem}
Classify the group of order $6$ by analyzing the following three cases: \\
\indent a). $G$ contains an element of order $6$. \\ 
\indent b). $G$ contains an element of order $3$ but not of order $6$. \\
\indent c). All elements of $G$ have order $1$ or $2$.

\end{problem}

\textbf{Proof:} $Z_{6}$ is a group of order $6$ that contains an element of order $6$. $S_{3}$ is a group of order $6$ that contains order $3$ but not of order $6$. Suppose we have a group of order $6$ where all elements of $G$ have order $1$ or $2$, say the group $G = \{a_{1}, a_{2}, a_{3}, a_{4}, a_{5}, a_{6}\}$, then only one of them can have order $1$, which is essentially $e$, and we let it be $a_{1}$, then for $i=2, 3, 4, 5, 6$, $a_{i}$ has order 2, which implies that $a_{i} = a_{i}^{-1}$. We assume that $a_{2}a_{3} = a_{4}$, then $a_{4}a_{5}=a_{6}$, otherwise if $a_{4}a_{5} = a_{2}$, then $a_{2}^{-1}a_{4} = a_{5}^{-1}$, which means $a_{2}a_{4}=a_{5}$, however, $a_{2}a_{4} = a_{3}$, then we would have $a_{3} = a_{5}$, which is absurd. $a_{4}a_{5}\neq a_{3}$, $a_{4}a_{5}\neq a_{4}$, and $a_{4}a_{5}\neq a_{5}$ follows the same reasoning. Consider $a_{2}a_{5}=a_{3}a_{4}a_{5}=a_{3}a_{6}$, then this product can only be $a_{4}$, which means that $a_{2}a_{5}=a_{2}a_{3}$, then $a_{5}=a_{3}$, which is absurd. Thus the c) case cannot be possible. To conclude, a group of order $6$ can be classified as either isomorphic to $Z_{6}$ or $S_{3}$. 

\begin{problem}
Give an example of a finite group $G$ that satisfies the following both properties: \\
\indent a). $G$ is of order $8$. \\
\indent b). $G$ is not abelian.
\end{problem}

\textbf{Solution:} Consider the order 8 group of $\{e, r, r^2, r^3, f, fr, fr^2, fr^3\}$ where $f^2 = e$, $r^4 = e$, and $fr = r^3f$. Verify this is indeed a group, by calculation, it is closed under internal law, and for each element, there is one inverse ($r^{-1} = r^3$, $r^{-2} = r^{2}$, $f^{-1} = f^{1}$, $(fr)^{-1} = r^{3}f= fr$, $(fr^{2})^{-1} = fr^2$, $(fr^{3})^{-1} = (fr^3)$). Also, it is non-abelian since $r^3f = fr\neq fr^3$.

\end{document}
