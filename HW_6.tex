\documentclass[12pt]{article}
\usepackage[margin=1in]{geometry}
\usepackage[all]{xy}


\usepackage{amsmath,amsthm,amssymb,color,latexsym}
\usepackage{geometry}        
\geometry{letterpaper}    
\usepackage{graphicx}

\newcommand{\legendre}[2]{\ensuremath{\left( \frac{#1}{#2} \right) }}
\newtheorem{problem}{Problem}

\newenvironment{solution}[1][\it{Solution}]{\textbf{#1. } }{$\square$}


\begin{document}
\noindent Honors Algebra II \hfill Assignment 6\\
Yunzhe Zheng. (2025/04/02)

\hrulefill

\begin{problem}
Decompose $x^9-x$ into a product of monic irreducible polynomials over $\mathbb{F}_{3}$.
\end{problem}

\textbf{Solution:} $x^9-x=x(x^8-1)=x(x-1)(x^7+x^6+x^5+x^4+x^3+x^2+x+1)=x(x-1)(x+1)(x^2+1)(x^2+2x+2)(x^2+x+2)$ in $\mathbb{F}_3$.
\\
\begin{problem}
What is the product of all elements of the multiplicative group $\mathbb{F}_{29}^{*}.$
\end{problem}

\textbf{Solution:} Consider symmetric group $S_{29}$, since $29$ is prime, we notice that there are $28!$ elements of order $29$ (they are all of the form $(a_{1}, a_{2}, \dots, a_{29})$, and we can fix $a_1$ to be 1 and permute other entries, which gives us $28!$). Since for each $29$-group (automatically Sylow $29$-group), it has $29-1=28$ elements of order $29$ (excluding $1$), then there are $(p-2)!$ Sylow $29$-group in $S_{29}$, then by Sylow Third theorem, $(29-2)!\equiv 1\mod 29$, then $(29-1)!\equiv 28\mod 29$, which implies that product of all elements in $\mathbb{F}_{29}^*$ is $-1$.
\\
\begin{problem}
Find two first primes which remain primes in the rings of algebraic integers in the fields $\mathbb{Q}[i]$ and $\mathbb{Q}[s]$ where $s^2=-5$. 
\end{problem}

\textbf{Solution:} Since $i^2=-1\equiv 3\mod 4$ and $-5\equiv 3\mod 4$, then from Artin's Algebra Theorem 13.6.1(c), it is sufficient to find prime integer $p$ and check whether $-1$ and $-5$ are square modulo $p$, which is, by the Euler criterion, equivalent to checking $(-1)^{\frac{p-1}{2}}\equiv -1\mod p$ and $(-5)^{\frac{p-1}{2}}\equiv -1\mod p$. From calculation, the first two integer primes satisfying that equality are $11, 19$.
\\
\begin{problem}
Find all irreducible polynomials of degree $3$ over $\mathbb{F}_4$.
\end{problem}

\textbf{Solution:} $\mathbb{F}_4=\{0, 1, \alpha, \alpha+1\}$, and since polynomial of degree $3$ is irreducible if and only if it has no root in $\mathbb{F}_4$, by enumerating all possibilities on the coefficients, we obtain all such possibilities: $x^3+1$, $x^3+\alpha$, $x^3+\alpha+1$, $x^3+x+1$, $x^3+\alpha x+1$, $x^3+(\alpha+1)x+1$, $x^3+x^2+1$, $x^3+x^2+x+\alpha$, $x^3+x^2+\alpha x+\alpha+1$, $x^3+x^2+(\alpha+1)x+\alpha$, $x^3+\alpha x^2+1$, $x^3+\alpha x^2+x+\alpha+1$, $x^3+\alpha x^2+\alpha x+\alpha$, $x^3+\alpha x^2+(\alpha+1)x+\alpha$, $x^3+\alpha x^2+(\alpha+1)x+(\alpha+1)$, $x^3+(\alpha+1)x^2+1$, $x^3+(\alpha+1)x^2+x+\alpha$, $x^3+(\alpha+1)x^2+\alpha x+\alpha$, $x^3+(\alpha+1)x^2+\alpha x+\alpha+1$, $x^3+(\alpha+1)x^2+(\alpha+1)x+\alpha+1$.
\\
\begin{problem}
Find solvable subgroups of $S_6$ with transitive action on $6$ points. (As many as you can modulo isomorphism).
\end{problem}

\textbf{Solution:} 1. $C_{6}$ is a subgroup acting transitively on $6$ points which is obviosuly solvable because it is abelian.\\
\indent 2. $D_6\subseteq S_6$ acts transitively on $6$ points since it characterizes rotations and flips of a hexagon. Consider the sequence $D_6\supset C_6\supset \{e\}$, then $D_6/C_{6}$ has prime order, thus is abelian. $C_6$ is abelian, hence $D_6$ is solvable. \\
\indent 3. $S_{3}$ acts transitively on itself, which has $6$ points. It is solvable by considering the sequence $S_{3}\supseteq A_3\supseteq \{e\}$, where $S_3/A_3$ is abelian since has order $2$, and $A_3$ is abelian. \\
\indent 4. $A_4$ acts transitively on itself, which has $12$ elements, then by taking a subgroup of order $2$ in $A_4$, say $\langle (1\ 2)(3\ 4)\rangle$, we obtain $6$ left (right) coset of $A_4$, and surely $A_4$ act transitively on these cosets. Also, $A_4$ is solvable by considering a normal subgroup of order $4$ in $A_4$, namely $S:=\{(1\ 2)(3\ 4), (1\ 3)(2\ 4), (1\ 4)(2\ 3), \text{Id}\}$. Notice that $S$ itself is abelian, and $A_4/S$ is abelian since it has prime order.\\
\indent 5. $S_4$ acts tranitively on 6 points by the same reasoning as before, but pick a subgroup of order $4$ in $S_4$, say $\langle (1\ 2), (3\ 4)\rangle$. $S_4$ is also solvable by considering the similar sequence $S_4\supseteq A_4\supseteq S\supseteq\{e\}$, where $A_{4}$ is normal in $S_{4}$ because it has index $2$, and thus $S_{4}/A_{4}$ is abelian. \\
\end{document}