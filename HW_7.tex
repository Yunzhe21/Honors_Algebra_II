\documentclass[12pt]{article}
\usepackage[margin=1in]{geometry}
\usepackage[all]{xy}


\usepackage{amsmath,amsthm,amssymb,color,latexsym}
\usepackage{geometry}        
\geometry{letterpaper}    
\usepackage{graphicx}

\newcommand{\legendre}[2]{\ensuremath{\left( \frac{#1}{#2} \right) }}
\newtheorem{problem}{Problem}

\newenvironment{solution}[1][\it{Solution}]{\textbf{#1. } }{$\square$}


\begin{document}
\noindent Honors Algebra II \hfill Assignment 7\\
Yunzhe Zheng. (2025/04/16)

\hrulefill

\begin{problem}
Describle all possible complex $3$-dimensional complex representations of $\mathbb{Z}$ modulo equivalence.
\end{problem}

\textbf{Solution:} Since $\mathbb{Z}$ is generated by $1$, we only need to consider $\rho_1$ given representation $\rho: \mathbb{Z}\to GL_3(\mathbb{C})$. By Jordan Decomposition theorem, it's sufficient to categorize $3\times 3$ Jordan normal form as follows: \\
\indent 1. Diagonal matrix with nonzero distinct diagonal entries. $\begin{pmatrix}
    \lambda_1 && 0 && 0 \\
    0 && \lambda_2 && 0 \\
    0 && 0 && \lambda_3
\end{pmatrix}$, where $\lambda_i$ are nonzero and distinct.\\
\indent 2. Diagonal matrix with two identical diagonal entries. $\begin{pmatrix}
    \lambda_1 && 0 && 0 \\
    0 && \lambda_1 && 0 \\
    0 && 0 && \lambda_2
\end{pmatrix}$, where $\lambda_i$ are nonzero and distinct. \\
\indent 3. Diagonal matrix with identical diagonal entries. $\begin{pmatrix}
    \lambda_1 && 0 && 0 \\
    0 && \lambda_1 && 0 \\
    0 && 0 && \lambda_1
\end{pmatrix}$, where $\lambda_1$ nonzero.\\
\indent 4. One $2\times 2$ Jordan block and one $1\times 1$ block. $\begin{pmatrix}
    \lambda_1 && 1 && 0 \\
    0 && \lambda_1 && 0 \\
    0 && 0 && \lambda_2 
\end{pmatrix}$, where $\lambda_i$ are nonzero. \\
\indent 5. One $3\times 3$ Jordan block. $\begin{pmatrix}
    \lambda_1 && 1 && 0 \\
    0 &&\lambda_1 && 1 \\
    0 && 0 && \lambda_1
\end{pmatrix}$, where $\lambda_1$ nonzero.
\\
\begin{problem}
Let $G=\mathbb{Z}_n$ be a cyclic group of order $n$ coprime to $p$. Show that any linear representation of $G$ over a field $F$ of characteristic $p$ is a direct sum of irreducible representations. 
\end{problem}

\textbf{Proof:} For any representation $\rho: \mathbb{Z}_n\to GL(\mathbb{F}_p)$, only need to consider $\rho_1$ because $\mathbb{Z}_n$ is generated by $1$. Up to choosing a basis, we can write $\rho_1$ as matrix $T$, where $T^n=I$, the minimal polynomial of $A$ divides $x^n-1$, where $x^n-1=0$ has no multiple roots given $\gcd(n,p)=1$, then $T^n-1=f(T)g(T)$, $f$ is irreducible. Take $V=\ker (f(T))$, then it is an invariant subspace, thus the representation can be written as $\begin{pmatrix}
    A && B \\
    0 && C
\end{pmatrix}$, where $A,B,C$ are block matrices. Then it's sufficient to show that $B=0$. Since $T^n=I$, then $\begin{pmatrix}
    A && B \\
    0 && C
\end{pmatrix}^n=\begin{pmatrix}
    A^n && \sum_{k=0}^{n-1}{p \choose n} A^kBC^{n-k-1} \\
    0 && C^n
\end{pmatrix}$, then notice that in order to make summation $0$, we have to have $B=0$. Finally by doing the same on matrix $C$ and $A$, eventually we can obtain that any representation can be decomposed into direct sum of irreducible representation.
\\
\begin{problem}
What is going to be the tensor product of two copies of the representation $V_2$ of the group $S_3$.
\end{problem}

\textbf{Solution:} Representation $V_2$ is $\rho:S_3\to GL(\mathbb{R}^2)$, and we pick standard basis $\{e_1, e_2\}$ of $\mathbb{R}^2$. Let $x=(1\ 2\ 3)$ and $y=(1\ 2)$, then the matrix representations are:
$$
M_x=\begin{pmatrix}
\cos(2\pi/3) && -\sin(2\pi/3) \\ \sin(2\pi/3) && \cos(2\pi/3)
\end{pmatrix}, M_y=
\begin{pmatrix}
1 && 0 \\
0 && -1
\end{pmatrix}
$$
The basis of $\mathbb{R}^2\otimes\mathbb{R}^2$ is $\{e_1\otimes e_1, e_1\otimes e_2, e_2\otimes e_1, e_2\otimes e_2\}$, and by the definition of tensor product of two representation $\rho^1,\rho^2$, $\rho_s'(v_1\cdot v_2)=\rho^1_s(v_1)\cdot\rho^1_s(v_2)$. So we only need to compute for basis.
$$
\begin{aligned}
\rho_x'(e_1\otimes e_1)&=M_xe_1\otimes M_xe_1= \frac{1}{4}e_1\otimes e_1-\frac{\sqrt{3}}{4}e_1\otimes e_2-\frac{\sqrt{3}}{4}e_2\otimes e_1+\frac{3}{4}e_2\otimes e_2\\
\rho_x'(e_1\otimes e_2)&=M_xe_1\otimes M_xe_2=\frac{\sqrt{3}}{4}e_1\otimes e_1+\frac{1}{4}e_1\otimes e_2-\frac{3}{4}e_2\otimes e_1-\frac{\sqrt{3}}{4}e_2\otimes e_2 \\
\rho_x'(e_2\otimes e_1)&=M_xe_2\otimes M_xe_1=\frac{\sqrt{3}}{4}e_1\otimes e_1-\frac{3}{4}e_1\otimes e_2+\frac{1}{4}e_2\otimes e_1-\frac{\sqrt{3}}{4}e_2\otimes e_2 \\
\rho_x'(e_2\otimes e_2)&=M_xe_2\otimes M_xe_2= \frac{3}{4}e_1\otimes e_1+\frac{\sqrt{3}}{4}e_1\otimes e_2+\frac{\sqrt{3}}{4}e_2\otimes e_1+\frac{1}{4}e_2\otimes e_2\\
\end{aligned}
$$
then the result is 
$$
\begin{pmatrix}
\frac{1}{4} && \frac{\sqrt{3}}{4} && \frac{\sqrt{3}}{4} && \frac{3}{4}\\
-\frac{\sqrt{3}}{4} && \frac{1}{4} && -\frac{3}{4} && \frac{\sqrt{3}}{4}\\
-\frac{\sqrt{3}}{4} && -\frac{3}{4} && \frac{1}{4} && \frac{\sqrt{3}}{4} \\
\frac{3}{4} && -\frac{\sqrt{3}}{4} && -\frac{\sqrt{3}}{4} && \frac{1}{4} 
\end{pmatrix}
$$
the other one is 
$$
\begin{pmatrix}
1 && 0 && 0 && 0 \\
0 && -1 && 0 && 0 \\
0 && 0 && -1 && 0 \\
0 && 0 && 0 && 1
\end{pmatrix}
$$
\\ 
\begin{problem}
Show that $PSL(2,\mathbb{Z}_5)$ is a simple group (it is the quotient of $SL(2,\mathbb{Z}_5)$ modulo the central subgroup consisting of diagonal matrices).
\end{problem}

\textbf{Proof:} $PSL(2,\mathbb{Z}_5)\simeq A_5$, which is simple. 

\begin{problem}
Describe character table of elements of $A_4$.
\end{problem}

\textbf{Solution:} $A_4$ has four conjugacy classes, namely $C_1=[e]$, $C_2=[(1\ 2)(3\ 4)]$, $C_3=\{(1\ 2\ 3), (1\ 3\ 4), (1\ 4\ 2), (2\ 4\ 3)\}$ and $C_4=\{(1\ 3\ 2), (1\ 4\ 3), (1\ 2\ 4), (2\ 3\ 4)\}$. There are four isomorphism classes of irreducible representations, with dimensions satisfying $|A_4|=12=d_1^2+d_2^2+d_3^2+d_4^2$, the only solution to that is $d_1=d_2=d_3=1$, $d_4=3$. The characteristic table lookes like:
\begin{center}
\begin{tabular}{||c c c c c||} 
 \hline
  & $C_1$ & $C_2$ & $C_3$ & $C_4$ \\ [0.5ex] 
 \hline\hline
 $\chi_1$ & 1 & 1 & 1 & 1 \\ 
 \hline
 $\chi_2$ & 1 & a & b & c \\
 \hline
 $\chi_3$ & 1 & d & e & f \\
 \hline
 $\chi_4$ & 3 & g & h & i \\
 \hline
\end{tabular}
\end{center}
By orthogonal theorem, $\langle\chi_1,\chi_2\rangle = \frac{1}{12}(1+3a+4b+4c)=0$. Since $\chi_2(x)=a$ is the trace of a $1\times 1$ matrix, and $x^2=1$, then $a=\pm 1$. Similarly with $b,c$, the possible values are $1, \omega, \omega^2$ where $\omega$ is the third root of unity. However, $b$ or $c$ equals $1$ is impossible from the equation, and hence $a=1$. Same procedure for $\chi_3$. Finally for $\chi_4$, by checking $\langle\chi_1,\chi_4\rangle=0$, $\langle\chi_2,\chi_4\rangle=0$ and $\langle\chi_3,\chi_4\rangle=0$, we can obtain that $g=-1$, $h=i=0$. The complete character map is as follows:
\begin{center}
\begin{tabular}{||c c c c c||} 
 \hline
  & $C_1$ & $C_2$ & $C_3$ & $C_4$ \\ [0.5ex] 
 \hline\hline
 $\chi_1$ & 1 & 1 & 1 & 1 \\ 
 \hline
 $\chi_2$ & 1 & 1 & $\omega$ & $\omega^2$ \\
 \hline
 $\chi_3$ & 1 & 1 & $\omega^2$ & $\omega$ \\
 \hline
 $\chi_4$ & 3 & -1 & 0 & 0 \\
 \hline
\end{tabular}
\end{center}

\end{document}