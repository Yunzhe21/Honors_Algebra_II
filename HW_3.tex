\documentclass[12pt]{article}
\usepackage[margin=1in]{geometry}
\usepackage[all]{xy}


\usepackage{amsmath,amsthm,amssymb,color,latexsym}
\usepackage{geometry}        
\geometry{letterpaper}    
\usepackage{graphicx}

\newtheorem{problem}{Problem}

\newenvironment{solution}[1][\it{Solution}]{\textbf{#1. } }{$\square$}


\begin{document}
\noindent Honors Algebra II \hfill Assignment 3\\
Yunzhe Zheng. (2025/02/19)

\hrulefill

\begin{problem}
Describe the commutative Sylow subgroups of the permutation group $S_{20}$. (i.e. find all prime $p$ with the property that $Syl_{p}(S_{20})$ is commutative and describe corresponding Sylow groups).
\end{problem}

\textbf{Solution:} $|S_{20}|=2^{18}\times 3^{8}\times 5^{4}\times 7^{2}\times 11\times 13\times 17\times 19$. Since Sylow subgroups are commutative if and only if $n=20 < p^2$, then $p=5, 7, 11, 13, 17, 19$. When $p=11, 13, 17, 19$, the corresponding Sylow $p$-subgroups have prime order, and thus are cyclic. When $p=5$, the order of Sylow $5$-subgroup is of order $5^4$, then it must be of the form $C_{5}\times C_{5}\times C_{5}\times C_{5}$, since otherwise we must have a cycle with at least $25$ elements, exceeding $n = 20$. When $p=7$, the order of Sylow $7$-subgroup is $7^2$, then it is in the form of $C_{7}\times C_{7}$, otherwise it will be a cycle of $49$ elements. To conclude, they are of the following form $C_{11}, C_{13}, C_{17}, C_{19}, C_{5}\times C_{5}\times C_{5}\times C_{5}$, or $C_{7}\times C_{7}$.

\begin{problem}
Describe elements of maximum order in $S_{16}$.
\end{problem}

\textbf{Solution:} 
Finding elements of maximum order is essentially asking for a partition of $16$, say $c_{1},\dots, c_{n}$ such that $\sum\limits_{i=1}^{n}=16$, and maximize $\text{lcm}(c_{1}, \dots, c_{n})$. Consider all prime numbers and their power smaller or equal to $16$, then they are $\{2, 3, 5, 7, 11, 13, 4, 8, 16, 9\}$. We have the following possibilities $(2, 3, 4, 2, 5)$, $(2, 3, 4, 7)$, $(2, 3, 4, 3, 4)$, $(2, 3, 5, 6)$, $(2, 3, 5, 1, 5)$, $(2, 3, 5, 2, 4)$, $(2, 3, 7, 4)$, $(2, 3, 8, \dots)$, $(2, 3, 9, \dots)$, $(2, 3, 11)$, $(2, 4, 5, 2, 3)$, $(2, 4, 7, 3)$, $(2, 4, 8, 2)$, $(2, 5, 7, \dots)$, $(2, 5, 8, 1)$, $(2, 5, 9)$, $(2, 7, 7)$, $(2, 8, 6)$, $(2, 9, 5)$, $(2, 9, 1, 4)$, $(2, 11, 2)$, $(2, 13, 1)$, $(3, 4, 5, \dots)$, $(3, 5, 7, 1)$, $(3, 5, 8)$, $(3, 7, 6)$, $(3, 7, 2, 4)$, $(3, 8, 5)$, $(3, 8, 1, 4)$, $(3, 8, 2, 2)$, $(3, 9, 4)$, $(3, 9, 2, 2)$, $(3, 11, 2)$, $(3, 13)$, $(4, 5, 7)$, $(4, 5, 1, 6)$, $(4, 5, 2, 5)$, $(4, 5, 3, 4)$, $(4, 7, 1, 4)$, $(4, 7, 2, 3)$, $(4, 8, 1, 3)$, $(4, 8, 2, 2)$, $(4, 9, 3)$, $(4, 9, 1, 2)$, $(4, 11, 1)$, $(5, 7, 4)$, $(5, 7, 1, 3)$, $(5, 7, 2, 2)$, $(5, 8, 3)$, $(5, 8, 1, 2)$, $(5, 9, 2)$, $(5, 11)$, $(7, 8, 1)$, $(7, 9)$, $(8, 8)$, then the maximum order is $140=4\times 5\times 7$, then such element is in the form of $(a_{1}, a_{2}, a_{3}, a_{4})(a_{5}, a_{6}, a_{7}, a_{8}, a_{9})(a_{10}, a_{11}, a_{12}, a_{13}, a_{14}, a_{15}, a_{16})$. 

\begin{problem}
Describe the commutator subgroup of $Syl_{3}(S_{9})$
\end{problem}

\textbf{Solution:} $[Syl3(S9),Syl3(S9)]=Syl3(S3)=C3[Syl_{3}(S_{9}), Syl_{3}(S_{9})]=Syl_{3}(S_{3})=C_{3}$

\begin{problem}
Find the number of affine lines in affine plane $F_{5}^{2}$.
\end{problem}

\textbf{Solution:} the set of affine lines is $\{y=ax+b: a, b\in F_{5}\}\cup \{x=c: c\in F_{5}\}$, then the number is $5^2+5=30$.
\end{document}