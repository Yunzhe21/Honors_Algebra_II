\documentclass[12pt]{article}
\usepackage[margin=1in]{geometry}
\usepackage[all]{xy}


\usepackage{amsmath,amsthm,amssymb,color,latexsym}
\usepackage{geometry}        
\geometry{letterpaper}    
\usepackage{graphicx}

\newtheorem{problem}{Problem}

\newenvironment{solution}[1][\it{Solution}]{\textbf{#1. } }{$\square$}


\begin{document}
\noindent Honors Algebra II \hfill Assignment 2\\
Yunzhe Zheng. (2025/02/12)

\hrulefill

\begin{problem}
Find the number of Sylow $2$-subgroups in $S_{5}$. 
\end{problem}

\textbf{Solution:} Order of Sylow $2$-group in $S_{5}$ has order $2^3=8$. Notice that $D_{4}$ also has order $8$, and we consider four indices out of five as vertices of a rectangle, then $D_{4}$ can be represented as permutations. There are $5$ ways to pick four vertices, and within each choice, there are $4!=24$ ways to name each vertex, 6 distinct namings up to rotation, and finally $3$ distinct namings up to both flipping and rotation. Thus the number is $15$, given the fact that the only options are $3, 5, 15$ by Sylow Theorem.
\\
\begin{problem}
Show that finite group $G$ of cardinal $p^{\alpha}m$ with $(m,p)=1$ has a subgroup of order $p^{i}$ for any $i\leq \alpha$.
\end{problem}

\textbf{Proof:} Prove by induction, for the base case where $\alpha = 1$, the conclusion is straight from Sylow Theorem. Now suppose for $\alpha\leq n$, the conclusion is true, then for $n + 1$, by Sylow Theorem, there exists a Sylow $p$-group $S$, of order $p^{n+1}$. Since $p$-group has a non-trivial center, pick $x\in Z(S)$ such that $x$ has order $p^{k}$ for some $0<k\leq n + 1$. We raise $x$ by power $p^{k-1}$, then $x^{p^{k-1}}$ has order $p$. Now consider $S/\langle x^{p^{k-1}}\rangle$, which has order $p^n$, and by induction it has a subgroup of order $p^{i}$ for any $i\leq n=\alpha$, thus by considering the canonical homomorphism $S\to S/\langle x^{p^{k-1}}\rangle$, it's preimage is what we desired. \qed \\

\begin{problem}
Let $G$ be a group of order $255$. Show that $G$ is not simple.
\end{problem}

\textbf{Proof:} Notice that $255=3\times 5\times 17$. Consider the Sylow $17$-subgroup of $G$. By Sylow Theorem, the number of Sylow $17$-group is congruent to $1$ mod $17$, and it divides $3 \times 5=15$, only $1$ satisfies both conditions, thus there is only one Sylow $17$-subgroup in $G$, which is normal in $G$, hence $G$ is not simple. \qed

\end{document}
